
%%%%%%%%%%%%%%%%%%%%%%% file typeinst.tex %%%%%%%%%%%%%%%%%%%%%%%%%
%
% This is the LaTeX source for the instructions to authors using
% the LaTeX document class 'llncs.cls' for contributions to
% the Lecture Notes in Computer Sciences series.
% http://www.springer.com/lncs       Springer Heidelberg 2006/05/04
%
% It may be used as a template for your own input - copy it
% to a new file with a new name and use it as the basis
% for your article.
%
% NB: the document class 'llncs' has its own and detailed documentation, see
% ftp://ftp.springer.de/data/pubftp/pub/tex/latex/llncs/latex2e/llncsdoc.pdf
%
%%%%%%%%%%%%%%%%%%%%%%%%%%%%%%%%%%%%%%%%%%%%%%%%%%%%%%%%%%%%%%%%%%%


\documentclass[runningheads,a4paper]{llncs}
\usepackage{graphicx}
\usepackage{url}
\usepackage[listings]{tcolorbox}
\usepackage{amssymb}
\usepackage{pifont}

\newcommand{\critics}{{\small{\sc{Critics}}}}
\newcommand{\phabricator}{{\small{\sc{Phabricator}}}}
\newcommand{\gerrit}{{\small{\sc{Gerrit}}}}
\newcommand{\codeflow}{{\small{\sc{CodeFlow}}}}
\newcommand{\collaborator}{{\small{\sc{Collaborator}}}}
\newcommand{\clusterchanges}{{\small{\sc{ClusterChanges}}}}
\newcommand{\delCode}{\textcolor{black}}
\newcommand{\addCode}{\textcolor{black}}
\newcommand{\ttt}[1]{\tt\small{#1}}


% -----------------------------------------------------------------
% color
% -----------------------------------------------------------------
\definecolor{javared}{rgb}{0.6,0,0} % for strings
\definecolor{javagreen}{rgb}{0.25,0.5,0.35} % comments
\definecolor{javapurple}{rgb}{0.5,0,0.35} % keywords
\definecolor{javadocblue}{rgb}{0.25,0.35,0.75} % javadoc

% ===============================================
% MyJavaSmallStyle
% ===============================================
\lstdefinestyle{MyJavaSmallStyle} {
  language=Java,
  frame=none,
  xleftmargin=15pt, 
  stepnumber=1, 
  numbers=left, 
  numbersep=5pt,
  numberstyle=\tiny\color[gray]{0.777}, 
  belowcaptionskip=\bigskipamount,
  captionpos=b, 
  escapeinside={*'}{'*},
  tabsize=5,
  emphstyle={\bf},
  basicstyle=\scriptsize\ttfamily,
  keywordstyle=\color{javapurple}\bfseries,
  stringstyle=\color{javared},
  commentstyle=\color{javagreen},
  morecomment=[s][\color{javadocblue}]{/**}{*/},
  showspaces=false,
  columns=flexible,
  showstringspaces=false,
  morecomment=[l]{//},
  tabsize=2,
  morekeywords={, Package,Invariant,Class,Method,Field,Where,in,Assert,ToLc,Split,Msg,Immutable,<<<,eq,neq,not,has,Assert,AssertExists,Attribute,Uc,Lc,},
  breaklines=true
}

\usepackage{amssymb}
\setcounter{tocdepth}{3}
\usepackage{graphicx}

\usepackage{url}

\begin{document}

\mainmatter  % start of an individual contribution

% first the title is needed
\title{Software Evolution} 

% a short form should be given in case it is too long for the running head
\titlerunning{Lecture Notes in Computer Science: Authors' Instructions}

% the name(s) of the author(s) follow(s) next
%
% NB: Chinese authors should write their first names(s) in front of
% their surnames. This ensures that the names appear correctly in
% the running heads and the author index.
%
\author{Na Meng, Tianyi Zhang, Miryung Kim} 

\institute{Virginia Tech and University of California, Los Angeles} 


\toctitle{Handbook on Software Engineering} 
\tocauthor{Na Meng, Tianyi Zhang and Miryung Kim}
\maketitle


\begin{abstract}
	\todo{Miryung is in charge.} 
\end{abstract}


\section{Introduction}
\todo{2 page}
- explain the definition of software evolution (cite: belady and lehman, etc)
- describe why this chapter focus on code changes, rather than other types of artefacts such as requirements, specifications, design documents, etc. 
- argue why software evolution is important (cite: code decay, eick et al.) 

\section{Concepts and Principles}
\todo{4 page} 
- explain a broad category of changes: corrective, adaptive, and perfective changes (cite kemerer and slaugher)  
- include a diagram about the process of software evolution with focus on changes: (A) applying program changes (B) inspecting program changes, and (C) debugging and testing program changes to overview the rest of sections.  
- introduce topics under each of the three perspectives on program changes. 

\section{An Organized Tour of Seminal Papers from Three Perspectives} 

\subsection{Applying Program Changes}
\paragraph{Perfective Changes} 
	refactoring is a special category of automated transformation: refactoring practices (kim et al., emerson murphy hill, ralph johnson) 
\paragraph{Corrective Changes} 
	- discuss literature on bug fixes  
\paragraph{Additive Changes}
discuss literature on adding features, (systematic editing is a specialized technique for automated feature addition) source to source transformation 
\paragraph{Porting programs to different languages} 
\paragraph{Porting programs to different platforms} 

\subsection{Inspecting Program Changes}

\paragraph{Code Reviews Practices}
since code reviews is a common context where change inspection happens ( peter rigby)  code flow, collaborators in industry. generally starting from code review practices
\paragraph{Program Differencing} 
program differencing (going to back history of differencing, diff, AST diff, CFG diff, PDG diff.--- mention its symmetric problem of clone detection)
\paragraph{Techniques for Code Change Comprehension} 
go deeper for a few techniques that help with change comprehension: kim et al. critics, chris bird at MS


\subsection{Debugging and Testing Program Changes} 

\paragraph{Delta Debugging: Finding Code Changes Causing Errors}
\paragraph{Regression Testing} 
\paragraph{Change Impact Analysis} 

\section{Future Directions and Open Problems} 



\subsubsection*{Acknowledgments.} The heading should be treated as a
subsubsection heading and should not be assigned a number.

\section{The References Section}\label{references}
\bibliography{tianyi,mengna}
\bibliographystyle{abbrv}

\section*{Appendix} 
\end{document}
