Recorded change operations can be used to help programmers reason about software changes. Several editors or integrated development environment (IDE) extensions capture and replay keystrokes, editing operations, and high-level update commands to use the recorded change information for intelligent version merging, studies of programmers' activities, and automatic updates of client applications. When recorded change operations are used for helping programmers reason about software changes, this approach's limitation depends on the granularity of recorded changes. If an editor records only keystrokes and basic edit operations such as cut and paste, it is a programmer's responsibility to raise the abstraction level by grouping keystrokes. If an IDE records only high-level change commands such as refactorings, programmers cannot retrieve a complete change history. In general, capturing change operations to help programmers reason about software change is {\it impractical} as this approach constrains programmers to use a particular IDE.  Below, we discuss a few examples of recording change operations from IDEs:  

Spyware is a representative example in this line of work~\cite{Robbes2008:spyware}. It is a smalltalk IDE extension to capture AST-level change operations (creation, addition, removal and property change of an AST node) as well as refactorings. It captures refactorings during development sessions in an IDE rather than trying to infer refactorings from two program versions. The tool is used to study when and how programmers perform refactorings. MolhadoRef automatically resolves merging conflicts that a regular {\it diff}-based merging algorithm cannot resolve by taking into account the semantics of recorded move and rename refactorings~\cite{Dig2007}. CatchUp \cite{Henkel2005} captures API refactoring actions as a developer evolves an API and allows the users of the API to replay the refactorings to bring their client software up to date. 

%Evans et al. \cite{Evans2003} collected students' programming data by capturing keystroke, mouse and window focus events generated from the Windows operating system and used this data to observe programming practices. Similarly, Kim et al.~\cite{Kim04} studied copy and paste programming practices by recording keystrokes and edit operations in an Eclipse IDE. 
% granularity -> limitations 

