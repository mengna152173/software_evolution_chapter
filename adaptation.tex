Adaptive changes are applied to software, when its environment changes. In this section, we focus on three scenarios of adaptive changes: cross-system software porting, cross-language software migration, and software library upgrade (i.e., API evolution).

Consider an example of cross-system porting. When a software system is installed on a computer, the installation can depend on configurations of the hardware, the software, and the device drivers for particular devices. To make the software to run on a different processor or a operating system, and to make it compatible with different drivers, we may need adaptive changes to adjust the software to the new environment. 
Consider another example of cross-language migration where you have software in Java that must be translated to C. Developers need to rewrite software and must also update language-specific libraries.
Finally consider the example of API evolution. When the APIs of a library and a platform evolves, corresponding adaptations are often required for client applications to handle such API update. In extreme cases, e.g., when porting a Java desktop application to the iOS platform, developers need to rewrite everything from scratch, because both the programming language (i.e., Swift) and software libraries are different. 

\subsubsection{Cross-System Porting.} 
Software forking\textemdash creating a variant product by copying and modifying an existing product\textemdash is often considered an ad hoc, low cost alternative to principled product line development. To maintain such forked products, developers often need to port an existing feature or bug-fix from one product variant to another. 

\paragraph{{Empirical Studies on Cross-System Porting.}}
As OpenBSD, NetBSD, and FreeBSD have evolved from the same origin but been maintained independently from one another, many have studied the BSD family to investigate the extent and nature of cross-system porting. The studies found that 
(1) the information flow among the forked BSD family is decreasing according to change commit messages~\cite{Fischer2005}; 
(2) 40\% of lines of code were shared among the BSD family~\cite{Yamamoto2005}; 
(3) in some modules such as device driver modules, there is a significant amount of adopted code~\cite{Cordy2011:largecloning}; and
(4) contributors who port changes from other projects are highly active contributors according to textual analysis of change commit logs and mailing list communication logs~\cite{Canfora2011:bsdfork}. 

More recently, Ray et al.~comprehensively characterized the temporal, spatial, and developer dimensions of cross-system porting in the BSD family~\cite{Ray2012:FSE}. Their work computed the amount of edits that are ported from other projects as opposed to the amount of code duplication across projects, because not all code clones across different projects undergo similar changes during evolution, and similar changes are not confined to code clones. To identify ported edits, they compared the content and edit operations of program patches from 18 yeas of version history. Their study found that maintaining forked projects involves significant effort of porting patches from other projects and that cross-system porting is periodic and its rate does not necessarily decrease over time. A significant portion of active developers participate in porting changes from peer projects. Ported changes are less defect-prone than non-ported changes. 


%For example, Fischer et al.~analyzed change commit messages of the BSD family and found a decreasing trend of information flow between OpenBSD and other BSD projects~\cite{Fischer2005}. Yamamoto et al.~found that up to 40\% of lines of code were shared among the BSD family~\cite{Yamamoto2005}. James et al.~showed the evidence of adopted code in device driver modules between Linux and FreeBSD~\cite{Cordy2011:largecloning}. Canfora et al.~investigated the social characteristics of contributors who made cross-system bug fixes between FreeBSD and OpenBSD~\cite{Canfora2011:bsdfork} by using textual analysis of change commit logs and mailing list communication logs. They observed that contributors who port changes from other projects are highly active contributors. Ray et al.~comprehensively characterized the temporal, spatial, and developer dimensions of cross-system porting in the BSD family. Their study found that maintaining forked projects involves significant effort of porting patches from other projects and that cross-system porting  is periodic and its rate does not necessarily decrease over time~\cite{Ray2012:FSE}. %A significant portion of active developers participate in porting changes from peer projects. Ported changes are less defect-prone than non-ported changes. 

%\paragraph{{Cross-Platform Software Development.}}
%When assigning different software development teams to independently work on separate source trees targeting different platforms, different teams maintain almost identical functionality and incur redundant effort. Some programming languages, software libraries, and frameworks are built to facilitate cross-platform software development. With such tool support, developers only need to build software once, but generate executable software for multiple platforms.

 %(e.g.,Ultimate++~\cite{ultimate}) 
 %(e.g., cairo~\cite{cairo}) 
 %(e.g., 8th~\cite{8th})
%To simplify cross-platform mobile software development, existing tools support developers to write code in HTML5 (e.g., Sencha~\cite{sencha}), or even automatically translate HTML5 implementation to Android or iOS native code (e.g., PhoneGap~\cite{phonegap}).

\subsubsection{Cross-Language Migration.} 
When maintaining a legacy system that was written in an old programming language (e.g., Fortran) decades ago, programmers may migrate the system to a mainstream general-purpose language, such as Java, to facilitate the maintenance of existing codebase and to extend the system by leveraging new programming language features. 
%Different from the API adaptive changes mentioned above, such software translation requires of a significant amount of coding effort to rewrite the same application in a new programming language.
%\todo{Na, more techniques? I think this section is relatively weaker than others. I also do not think we can talk about TXL here, since we decided to talk about automated techniques separately.} 

\paragraph{{Cross-Language Program Translation.}}
To translate code implementation from one language to another, researchers have built tools by hard coding the translation rules and implementing any missing functionality between languages~\cite{Yasumatsu:95,1192409:03,Sneed:2010}. 
%SPiCE translates Smalltalk programs into a C environment by mapping the execution models of two languages. Specifically, 
Yasumatsu et al.~map compiled methods and contexts in Smalltalk to machine code and stack frames respectively, and implement runtime replacement classes in correspondence with the Smalltalk execution model and runtime system~\cite{Yasumatsu:95}. Mossienko~\cite{1192409:03} and Sneed~\cite{Sneed:2010} automate COBOL-to-Java code migration by defining and implementing rules to generate Java classes, methods, and packages from COBOL programs. 
{\em mppSMT} automatically infers and applies Java-to-C\# migration rules using a phrase-based statistical machine translation approach~\cite{7372046}. It encodes both Java and C\# source files into sequences of syntactic symbols, called {\em syntaxemes}, and then relies on the syntaxemes to align code and to train sequence-to-sequence translation. 

\paragraph{{Mining Cross-Language API Rules.}}
When migrating software to a different target language, API conversion poses a challenge for developers, because the diverse usage of API libraries induces an endless process of specifying API translation rules or identifying API mappings across different languages. Zhong et al.~\cite{zhong2010mining} and Nguyen et al.~\cite{nguyen2014statistical,Nguyen:2017:EAE} automatically mine API usage mappings between Java and C\#. Zhong et al.~align code based on similar names, and then construct the API transformation graphs for each pair of aligned statements~\cite{zhong2010mining}. StaMiner~\cite{nguyen2014statistical} mines API usage sequence mappings by conducting program dependency analysis~\cite{Muchnick:1998} and representing API usage as a graph-based model~\cite{Nguyen09}. %API2VEC mine API usage sequence mappings by converting API element sequences to vectors, and comparing cosine similarities between vectors~\cite{Nguyen:2017:EAE}. 

\subsubsection{Library Upgrade and API Evolution.}
Instead of building software from scratch, developers often use existing frameworks or third-party libraries to reuse well-implemented and tested functionality. Ideally, the APIs (Application Programming Interfaces) of libraries must remain stable such that library upgrades do not incur corresponding changes in client applications. In reality, however, APIs change their input and output signatures, change semantics, or are even deprecated, forcing client application developers to make corresponding adaptive changes in their applications.  

\paragraph{{Empirical Studies of API Evolution.}}
Dig and Johnson manually investigated API changes using the change logs and release notes to study the types of library-side updates that break compatibility with existing client code, and discovered that 80\% of such changes are refactorings~\cite{Dig2005}. Xing and Stroulia used UMLDiff to study API evolution and found that about 70\% of structural changes are refactorings~\cite{Xing2006:apievol}. Yokomori et al. investigated the impact of library evolution on client code applications using component ranking measurements~\cite{Yokomori2009:apiimpact}. Padioleau et al. found that API changes in the Linux kernel led to subsequent changes on dependent drivers, and such collateral evolution could introduce bugs into previously mature code~\cite{Padioleau2006:collateral}. McDonelle et al. examined the relationship between API stability and the degree of adoption measured in propagation and lagging time in the Android Ecosystem~\cite{McDonnell2013:api}. Hou and Yao studied the Java API documentation and found that a stable architecture played an important role in supporting the smooth evolution of the AWT/Swing API~\cite{Hou2011:api}. In a large scale study of the Smalltalk development communities, Robbes et al.~found that only 14\% of deprecated methods produce non-trivial API change effects in at least one client-side project; however, these effects vary greatly in magnitude. On average, a single API deprecation resulted in 5 broken projects, while the largest caused 79 projects and 132 packages to break~\cite{robbes2012}.

\paragraph{{Tool Support for API Evolution and Client Adaptation.}} 
Several existing approaches semi-automate or automate client adaptations to cope with evolving libraries.  Chow and Notkin~\cite{Chow1996} propose a method for changing client applications in response to library changes\textemdash a library maintainer annotates changed functions with rules that are used to generate tools that update client applications. Henkel and Diwan's CatchUp records and stores refactorings in an XML file that can be replayed to update client code~\cite{Henkel2005}. However, its update support is limited to three refactorings: renaming operations (e.g.  types, methods, fields), moving operations (e.g. classes to different packages, static members), or change operations (e.g. types, signatures). The key idea of CatchUp, {\em record-and-replay}, assumes that the adaptation changes in client code are exact or similar to the changes in the library side. Thus, it works well for replaying rename or move refactorings or supporting API usage adaptations via inheritance. However, CatchUp cannot suggest programmers how to manipulate the context of API usages in client code such as the surrounding control structure or the ordering between method-calls. Furthermore, CatchUp requires that library and client application developers use the same development environment to record API-level refactorings, limiting its adoption in practice. Xing and Stroulia's Diff-CatchU automatically recognizes API changes of the reused framework and suggests plausible replacements to the obsolete APIs based on the working examples of the framework codebasep~\cite{Xing2007:diffcatchup}. Dig et al.'s MolhadoRef uses recorded API-level refactorings to resolve merge conflicts that stem from refactorings; this technique can be used for adapting client applications in case of simple rename and move refactorings occurred in a library~\cite{Dig2007}.  

SemDiff~\cite{Dagenais2008:RAC} mines API usage changes from other client applications or the library itself.  It defines an adaptation pattern as a frequent {\em replacement} of a method invocation. That is, if a method call to $A.m$ is changed to $B.n$ in several adaptations, $B.n$ is likely to be a correct replacement for the calls to $A.m$. As SemDiff models API usages in terms of method calls, it cannot support complex adaptations involving multiple objects and method calls that require the knowledge of the surrounding context of those calls. LibSync helps client applications migrate library API usages by learning migration patterns~\cite{Nguyen2010:GAA} with respect to a partial AST with containment and data dependences. Though it suggests what code locations to examine and shows example API updates, it is {unable} to transform code automatically. Cossette and Walker found that, while most broken code may be mended using one or more of these techniques, each is ineffective when used in isolation~\cite{cossette2012}. 

% \paragraph{{API Usage Specification Extraction.}} 
%Several approaches extract API usage specifications from existing code. The forms of recovered API usage specifications and patterns include finite state automaton~\cite{zeller07,doc2spec}, pairs of method calls~\cite{Livshits2005,williams-tse05}, partial orders of calls~\cite{mapo-fse07,taoxie-ase09}, and Computation Tree Logic formulas~\cite{zeller-ase09}. The API usage representations in those static approaches are still limited, for example, the patterns are without control structures and involve only individual objects belonging to one class. In contrast to those static approaches, dynamic approaches recover the specifications by investigating the execution traces of programs but they require a huge amount of execution traces.  
%\cite{javert,perracotta,shoham-issta07,ramanathan-isce07,mike-ase09}


 
